\documentclass[a4paper]{article}

% Support for Norwegian letters
\usepackage[utf8x]{inputenc}
\usepackage[T1]{fontenc}
\usepackage{amsmath,amsfonts,amssymb,amsthm,booktabs,array,mathtools}
% consider package mhchem for typesetting chemical formulas

% Proper space and font for integral differential term
\newcommand{\dd}{\; \mathrm{d}} 
% Shorcut for ODEs with proper font
\newcommand{\diff}[2]{\frac{\mathrm{d} #1}{\mathrm{d} #2}}
% Shortcut for PDEs with proper font (shortcut: PDB)
\newcommand{\pdiff}[2]{\frac{\partial #1}{\partial #2}}
\newcommand{\pdiffn}[3]{\frac{\partial^{#3} #1}{\partial #2^{#3}}}

% Absolute value and norm commands.
% Read the mathtools.pdf to fix these!
\providecommand{\abs}[1]{\lvert#1\rvert} 

% Set the depth of section numbering
\setcounter{secnumdepth}{0}

\begin{document} 


\section{Materials and Methods}

All data analysis is performed with in house Python scipts. Statistics libraries
are either from scipy or from R accessed through the Python module rpy. Unless
otherwise noted, Spearman rank correlation coefficents are used.

\subsubsection{Calculating RNA-DNA and DNA-DNA hybrid energies}

RNA-DNA and DNA-DNA hybrid energies are calculated using a nearest neighbor
models that use the energy parameters from \cite{wu_temperature_2002-1} and
\cite{santalucia_thermodynamics_2004}, respectively.

\section{Results}

\subsubsection{Correlation of RNA-DNA and DNA-DNA binding energies with promoter
escape}

Figure X shows the change in the correlation
coefficient for the association between RNA-DNA energy and PY and DNA-DNA
energy and PY, when an increasing length of the ITR is taken into account. The
RNA-DNA energy vs. PY reaches significance at the 0.05 level at 10 nts and
levels out at 15 nts with p = 0.005. The DNA-DNA energy vs. PY association never
reaches statistical significance. Figure XXX shows a scatterplot of 15 nt
RNA-DNA energies vs PY. A strong RNA-DNA bond is associated with low rates of
promoter escape, while weaker RNA-DNA bonds may or may not give higher values of
promoter escape.

\subsubsection{Combining $\Delta G_{\text{RNA-DNA}}$ and $\Delta
G_{\text{DNA-DNA}}$ in a more physical model}

The 15nt RNA-DNA hybrid that gives peak
significance in Figure X is unphysical, in the sense that the length of the
RNA-DNA hybrid is at most 8-9 nt, yet it is a measure of the overall strength of
all the intermediate RNA-DNA hybrids formed inside RNAP during the initiation of
transcription. Although the DNA-DNA energy could not explain promoter escape
alone, it is possible that promoter escape can best be described as some function
of both $\Delta G_{\text{RNA-DNA}}$ and $\Delta G_{\text{DNA-DNA}}$. In
thermodynamic models of transcriptional elongation it is often assumed that
$\Delta G_{\text{RNA-DNA}}$, $\Delta G_{\text{DNA-DNA}}$, and $\Delta
G_{\text{RNA-RNA}}$ determine the state transitions of the elongating complex
\cite{tadigotla_thermodynamic_2006}, where $\Delta G_{\text{RNA-RNA}}$ is the
free folding energy of the nascent RNA closest to the elongating complex.
By further assuming that the elongating complex reaches local equilibrium
between each nucleotide addition \cite{greive_thinking_2005}, the rate of
transcription elongation can be written as proportional to the exponent of the
sum of the three free energy terms mentioned above:
\begin{equation}
	\label{eq:equilibrium1}
	r_e \propto e^{\Delta G_{\text{RNA-DNA}}+\Delta
	G_{\text{DNA-DNA}}+\Delta G_{\text{RNA-RNA}}},
\end{equation}
During promoter escape the nascent RNA extrudes no more than 5 nts out of the
RNA exit channel, making it unlikely that RNA secondary structures play a role in
promoter escape, so that the $\Delta G_{\text{RNA-RNA}}$ can be taken out of
equation \eqref{eq:equilibrium1} to give:
\begin{equation}
	\label{eq:equilibrium2}
	r_e \propto e^{\Delta G_{\text{RNA-DNA}}+\Delta
	G_{\text{DNA-DNA}}}.
\end{equation}
Indeed, a model that uses the energy terms in \eqref{eq:equilibrium2} to
predict the kinetic behavior of transcription initiation has been published
\cite{xue_kinetic_2008}, where the authors argue that the RNAP-$\Sigma$
holoenzyme also reaches local equilibrium between each nucleotide addition in
transcription initiation. If the local equilibrium assumption is true for
transcription initiation, and if $\Delta G_{\text{DNA-DNA}}$ matters at all to
this process, then the sum of $\Delta G_{\text{RNA-DNA}}$ and $\Delta
G_{\text{DNA-DNA}}$ should explain promoter escape better than $\Delta
G_{\text{RNA-DNA}}$ alone. To investigate this, we correlated $\Delta
G_{\text{RNA-DNA}}$+$\Delta G_{\text{DNA-DNA}}$ to promoter escape in a model that
takes into account the joint kinetic behaviour of the DNA bubble and the nascent
RNA during transcription initiation. In this model, it is assumed
that when RNAP initiates the open complex, DNA is unwound from nt -11 to +2
\cite{borukhov_rna_2008}. It is also assumed that transcription initiation works by
scrunching, i.e. the DNA bubble stays open from -11 and increases in size
downstream until promoter escape \cite{revyakin_abortive_2006}.

In this model, after the DNA bubble has been formed, the first nucleotide is added to form the
+1 RNA-DNA hybrid. Sequentially, the DNA bubble is increased by one nt
downstream, and the next nucleotide is added added to position +1. If $\Delta
G_{\text{DNA-DNA}}^{i,j}$ and $\Delta G_{\text{RNA-DNA}}^{i,j}$ denote the
DNA-DNA and RNA-DNA hybrid energies, respectively, from position i to j, then
the sum of sums of RNA-DNA and DNA-DNA energies are calculated as in equation
\eqref{eq:physicalsum} 
\begin{align}
	\label{eq:physicalsum}
	\sum \Delta G &=
	\left(\Delta G_{\text{DNA-DNA}}^{-11,2} + \Delta
	G_{\text{RNA-DNA}}^{1,1}\right) + \left(\Delta G_{\text{DNA-DNA}}^{-11,3} + \Delta
	G_{\text{RNA-DNA}}^{1,2}\right) + \nonumber \\
	 \dots + & \left(\Delta G_{\text{DNA-DNA}}^{-11,8} + \Delta G_{\text{RNA-DNA}}^{1,8}\right)
	+ \left(\Delta G_{\text{DNA-DNA}}^{-11,9} + \Delta
	G_{\text{RNA-DNA}}^{2,9}\right) +\nonumber \\
	 & \dots
	+ \left(\Delta G_{\text{DNA-DNA}}^{-11,21} + \Delta G_{\text{RNA-DNA}}^{13,20}\right).
\end{align}
As can be seen in FIGURE XXX, the sum of RNA-DNA and DNA-DNA energies is less
correlated with promoter escape than RNA-DNA energy alone.

By adding $\Delta G_{\text{RNA-DNA}}$ and $\Delta G_{\text{DNA-DNA}}$ (opposed
to subtracting them) it is assumed that both factors
act in the same directon; here, it is assumed that weak DNA-DNA and
RNA-DNA bonds better facilitate promoter escape than strong bonds. If instead
weak RNA-DNA bonds and strong DNA-DNA bonds facilitate promoter escape, the sum
of subtractions of $\Delta G_{\text{RNA-DNA}}$ and $\Delta G_{\text{DNA-DNA}}$
is given in \eqref{eq:minphysicalsum}.
\begin{align}
	\label{eq:minphysicalsum}
	\sum \Delta G &=
	\left(\Delta G_{\text{DNA-DNA}}^{-11,2} - \Delta
	G_{\text{RNA-DNA}}^{1,1}\right) + \left(\Delta G_{\text{DNA-DNA}}^{-11,3} - \Delta
	G_{\text{RNA-DNA}}^{1,2}\right) + \nonumber \\
	 \dots + & \left(\Delta G_{\text{DNA-DNA}}^{-11,8} - \Delta G_{\text{RNA-DNA}}^{1,8}\right)
	+ \left(\Delta G_{\text{DNA-DNA}}^{-11,9} - \Delta
	G_{\text{RNA-DNA}}^{2,9}\right) +\nonumber \\
	 & \dots
	+ \left(\Delta G_{\text{DNA-DNA}}^{-11,21} - \Delta G_{\text{RNA-DNA}}^{13,20}\right).
\end{align}
However, subtracting $\Delta G_{\text{RNA-DNA}}$ and $\Delta G_{\text{DNA-DNA}}$
in this way did not improve the correlation to promoter escape. 

\subsubsection{PolyA, polyT, and $\sigma^{70}$ binding sites}
%It is possible that polyA or polyT stretches in the UTR could cause
%transcriptional slippage CITATION, which can lead to either an aborted product or a too
%long finished transcript product; either way transcriptional slippage affects
%promoter escape and it could be possible to observe a relationship between polyA
%or polY stretches in the ITR of a promoter and how well it undergoes promoter
%escape. Figure XXX shows a table of the 43 ITR sequences, sorted by their productive
%yield value, where the columns show if a polyA or polyT stretch has been found,
%and if so, where it is located. Also in Figure XXX is a scatterplot of RNA-DNA
%energies vs promoter escape that illustrates what sequences have polyA or polT
%stretches and how many they have. There is no clear correlation between polyA or
%polyT stretches in the ITR of a promoter and its ability for promoter escape. If
%anything, there is a weak positive association between polyA and polyT stretches
%and promoter escape, likely because polyA and polyT sequences have weak RNA-DNA
%bonds.

Recent evidence indicates that $\Sigma^{70}$ is stochastically released from
RNAP after transcription initiation. If retained, $\sigma^{70}$ is
capable of rebinding the -10 like elements, causing transcriptional pausing
anywhere during transcription elongation \cite{mooney_tethering_2003}.
Specifically, $\sigma^{70}$ has been shown to induce pausing in the ITR region
of the $\lambda$ late gene operon \cite{ring_function_1996} though the
recognition of a -10 like element. The pause at this promoter lasts in the order
of a minute and therefore severely affects promoter escape.

Since a -10 like equence is likely to occur by chance in 43 randomized ITR
sequences and might affect promoter escape by causing $\sigma^{70}$-induced
pauses, we looked for an association between the occurence of -10 like sequences
in the ITR region of a promoter and its PY. In Figure XXX, the list of the 43
promoter ITRs are given with a column indicating if a minus 10 element has been
found in the ITR.  The -10 elements are either curated from literature or taken
from the RegulonDB database. As can be seen, there is no correlation between the
occurrence of -10 like elements in the ITR and PY. This shows that it is not
enough to have the presence of a -10 like element in the ITR to induce promoter
proximal pausing. However, Hsu has observed that the insertion of a -10
consensus sequence in the ITR has no affect on the abortive ladder on the XXX
promoter (Hsu, personal communcation).

% TODO Tomorrow, CONTINUE TO fill in all the citations and make the text
% correct. But first, skim those two papers found by scopus. You have read about
% the GGGC regulator in subtilus. Apparently, a pause before the C insertion can
% let the GGG slip upstream relative to the DNA template to allow another G to
% be added. Questions: 1)can it slip forward? 2) Will this leave the G
% dangeling? Or is it bound to the -1 element???? Do the -1,-2,-3 elements
% matter for backsliding in these slippage experiments? Can you check this for
% the experiments in question?
% After that, redo that minus10 finding like you suggested in Transcription.
% RESULT from BMC paper: very hard to improve promoter in the -1 to -40 area.
% Golden chance for ITR :)

\bibliographystyle{plain}
\bibliography{/home/jorgsk/phdproject/bibtex/jorgsk}	

\end{document}
