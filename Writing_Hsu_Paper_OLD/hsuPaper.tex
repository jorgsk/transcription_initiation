\documentclass[a4paper]{article}

% Support for Norwegian letters
\usepackage[utf8x]{inputenc}
\usepackage[T1]{fontenc}
\usepackage{amsmath,amsfonts,amssymb,amsthm,booktabs,array,mathtools}
% consider package mhchem for typesetting chemical formulas

% Proper space and font for integral differential term
\newcommand{\dd}{\; \mathrm{d}} 
% Shorcut for ODEs with proper font
\newcommand{\diff}[2]{\frac{\mathrm{d} #1}{\mathrm{d} #2}}
% Shortcut for PDEs with proper font (shortcut: PDB)
\newcommand{\pdiff}[2]{\frac{\partial #1}{\partial #2}}
\newcommand{\pdiffn}[3]{\frac{\partial^{#3} #1}{\partial #2^{#3}}}

% Absolute value and norm commands.
% Read the mathtools.pdf to fix these!
\providecommand{\abs}[1]{\lvert#1\rvert} 
\providecommand{\norm}[1]{\lVert#1\rVert}

% Set the depth of section numbering
\setcounter{secnumdepth}{0}

\begin{document} 
\subsection{Methods}
\begin{itemize}
	\item Explain about MSAT and the correction for msat by adding expected
		energies.
	\item Explain procedures for calculating correlation between abortive
		probabilities and RNA/DNA bond
	\item Procedure for calculating DNA/DNA bond and RNA/DNA+DNA/DNA bond.
	\item Procedure for correlating AP values to RNA/DNA bonds.
	\item Procedure for correlating AP values to NOTE: DNA/DNA bonds.
\end{itemize}
\subsection{Results}
\subsubsection{DNA/DNA bonds}
\begin{itemize}
	\item DNA/DNA bond does not explain PY values (show lack of correlation for
		increasing ITR length for both physical and unphysical model). FIGURE
		OK.
\end{itemize}
\subsubsection{RNA/DNA bonds}
\begin{itemize}
	\item Correlation exists between RNA/DNA bond of ITR and PY. ONLY NUMBER.
	\item Correlation increases almost monotonically with increasing ITR-length.
		\begin{itemize}
			\item The more information we have, the more we can explain. FIGURE
				OK.
		\end{itemize}
	\item Although correlation non-significant below nt 10, the near-monotinical
		increase is highly significant (probability of similar increase pattern
		occuring with random sequences is XXX). NUMBER OK.
	\item A 20nt RNA/DNA bond is unphysical. Calculation with physical RNA/DNA
		model (1,2), (1,3),..,(1,8/9),(2,9/10)..(11/12,20) revealed a similar
		correlation pattern. FIGURE OK.
	\item One plot of data with standard deviations. Suggestion: the best full
		ITR with E(energy) added after msat. FIGURE NOT PRESENT. SHOULD USE?
	\item Effect of outliers. N25/A1 big outlier. The rest not so bad. SEE ON
		FIGURE.	
	\item Correlation between purines and binding energy, show ladder. Show mean
		and stadard deviation for 43 runs of random sequences with normal
		distribution figure :). NUMBER OK.
\end{itemize}
\subsubsection{RNA/DNA+DNA/DNA}
\begin{itemize}
	\item Elongation models use a sum of RNA/DNA and DNA/DNA energies (++). A
		kinetic model of translation initiaton did the same. No correlation is
		found with both unphysical and physical methods. FIGURE OK.
\end{itemize}
\subsubsection{Abortive probabilities}
UNSURE IF I SHOULD INCLUDE THIS INFORMATION
\begin{itemize}
	\item AP values in general do not correlate with RNA/DNA values
	\item AP values in general do not correlate with DNA/DNA values
	\item However, looking at max/min AP, correlation was found.
\end{itemize}

\subsection{Discussion}
\begin{itemize}
	\item Weak RNA/DNA can cause transcriptional pausing (eg. Hippel
		``Transcriptional pausing caught in the act''), yet in the initiation of
		transcription weak RNA/DNA bonds correlate with high transcript amount,
		indicating that escape from promoter is more important than
		transcriptional pausing in the initiation of transcription -- because it
		is known with the lambda regulator that transcriptional pausing can
		affect early transcription.
	\item Hopothesis that DNA/DNA energies control abortive initiation. No role
		was found. WHO PUT FORTH THIS HYPOTHESIS?
	\item Hypothesis that AP-values can be predicted. In general no, except for
		the most extreme. 2008 kinetic model paper.
	\item Intrestingly results are in some cases better without correcting for
		msat, but within the standard deviation as calculated by putting random
		sequences after msat.
	\item Outliers: N25/A1 a naturally evolved promoter.
	\item Correlation found for max/min AP. Could be a result of large
		experimental errors, thus the incentive for max/min values. 
	\item  Argue that the energy-purine correlation is nothing outside what
		would be expected from random sequences. This shows that the string
		(0.77) purine-PY correlation in the data acts through a mechanism
		different from RNA/DNA bonds. What would have been the case if the
		RNA-purine link had been significantly larger/smaller than 0.31????
\end{itemize}


\end{document}



