%\documentclass[a4paper, draft]{article}
\documentclass[a4paper]{article}

% A new package for bibliography
% Use biber for bibtex and pdflatex instead of latex
\usepackage[backend=biber]{biblatex}
% new from biblatex
\addbibresource{/home/jorgsk/phdproject/bibtex/jorgsk.bib}

% Support for unicode
\usepackage[utf8]{inputenc}
\usepackage[T1]{fontenc}

\usepackage{amsmath,amsfonts,amssymb,amsthm,booktabs,array,mathtools,mhchem}
% consider package mhchem for typesetting chemical formulas

% Proper space and font for integral differential term
\newcommand{\te}[1]{\text{#1}} 
\newcommand{\dd}{\; \mathrm{d}} 
% Shorcut for ODEs with proper font
\newcommand{\diff}[2]{\frac{\mathrm{d} #1}{\mathrm{d} #2}}
% Shortcut for PDEs with proper font (shortcut: PDB)
\newcommand{\pdiff}[2]{\frac{\partial #1}{\partial #2}}
\newcommand{\pdiffn}[3]{\frac{\partial^{#3} #1}{\partial #2^{#3}}}

% Absolute value and norm commands.
% Read the mathtools.pdf to fix these!
\providecommand{\abs}[1]{\lvert#1\rvert} 
\providecommand{\norm}[1]{\lVert#1\rVert}

% Set the depth of section numbering
\setcounter{secnumdepth}{0}

\begin{document} 
\noindent

\subsection{Background}
We have found a strong and consistent correlation $(r ~ 0.8, p < 0.0001)$
between the productive yield ($PY$) and the parameters $\Delta
G_{\text{RNA-DNA}}$ and $Keq$. The correlation we found was
on the following form:
\\
\begin{equation}
	PY \sim \Delta G_{\text{RNA-DNA}} - Keq
	\label{eq:FRS}
\end{equation}
\\
We found this relationship almost by accident. Now  we need to ascertain the
physical meaning of this relationship.

A clue can be found in the scatterplot: there is an exponential relationship between
the right and left side of equation \ref{eq:FRS}. Thus, we are looking for a
term on the form
\begin{equation}
	PY \sim e^{\Delta G_{\text{RNA-DNA}} - Keq}
	\label{eq:goal}
\end{equation}

\subsection{Chemical equation for RNAP movement}

Below is a reaction equation for the RNA polymerase as it is translocating on DNA.

\begin{itemize}
	\item $\te{R}_i^{\te{e}}$ is the pre-translocated polymerase at nucleotide $i$
	\item $\te{R}_{i+1}^{\te{e}}$ is the pre-translocated polymerase at nucleotide $i+1$
	\item $\te{R}_i^{\te{s}}$ is the post-translocated polymerase at nucleotide $i$
	\item $\te{N}_{\te{t}}$ is the incoming NTP
	\item $\te{PP}_i$ is the released $\te{PP}_i$ from nucleotide incorporation
\end{itemize}
In the reaction equation RNAP osscillates between the pre-translocated and the
post-translocated step at some nucleotide position $i$. In the
post-translocated step, a nucleotide $\te{N}_{\te{t}}$ may bind in the active
site. This nucleotide might then be incorporated, bringing RNAP to position
$i+1$ and at the same time releasing $\te{PP}_i$.
\\
\begin{align}
\dots\ce{R_i^{\text{e}}
<=>
R_i^{\text{s}}
<=>
R_i^{\text{s}}N_t
<=>
R_{i+1}^{\text{e}}
+ PP_i\dots
}
\label{eq:full}
\end{align}
\\
This reaction ignores all pathways related to backtracking and abortion.

By skipping the nucleotide binding step and assuming that translocation is at
equilibrium, we obtain the following simplified equation:
\\
\begin{align}
\dots\ce{R_i^{\text{e}}
<=>[\ce{Keq_i}]
R_i^{\text{s}}
->[\ce{k1_i}]
R_{i+1}^{\text{e}}\dots
}
\label{eq:reduced3}
\end{align}
\\
Here ($Keq_i$) is the equilibrium constant for the translocation step:
\begin{equation}
	\te{R}_i^{\te{s}} = \frac{\te{R}_i^{\te{e}}}{Keq_i}
	\label{eq:keq}
\end{equation}
Also the reaction rate $k_1$ is the rate of the nucleotide incorporation step:

\subsection{From chemical equation to differential equation}
By assuming mass action kinetics, and using
\begin{equation*}
	k_1 = Ke^{-\frac{\Delta G}{RT}},
\end{equation*}
we arrive at the following expression:
\begin{equation*}
	\diff{\te{R}_{i+1}^{\te{e}}}{t} = Ke^{-\frac{\Delta G}{RT}}\te{R}_{i}^{\te{s}}.
\end{equation*}
Inserting for equation \ref{eq:keq}, we get
\begin{equation*}
	\diff{\te{R}_{i+1}^{\te{e}}}{t} = Ke^{-\frac{\Delta G}{RT}}\frac{\te{R}_i^{\te{e}}}{Keq_i}
\end{equation*}
By using
\begin{equation*}
	Keq_i = e^{\te{ln}(Keq_i)}
\end{equation*}
we obtain
\begin{align*}
	\frac{k_1}{Keq} &= \frac{Ke^{-\frac{\Delta G}{RT}}}{e^{\te{ln}(Keq)}}\\
	                &= Ke^{-\frac{\Delta G}{RT} - \te{ln}(Keq)}
\end{align*}
Thus 
\begin{equation}
	\diff{\te{R}_{i+1}^{\te{e}}}{t} \sim e^{-\frac{\Delta G}{RT} - \te{ln}(Keq)}
	\label{eq:stuff}
\end{equation}
If we assume that $\Delta G$ from the rate of nucleotide incorporation is
similar to $\Delta G_{\text{RNA-DNA}}$. Further the term $RT$ evaluates to
0.62. Thus we have 
\begin{equation*}
	\diff{\te{R}_{i+1}^{\te{e}}}{t} \sim e^{-1.6\Delta G_{\text{RNA-DNA}} - \te{ln}(Keq)}
\end{equation*}
which is a term very similar to what we started out with, namely:
\begin{equation*}
	e^{\Delta G_{\text{RNA-DNA}} - Keq}
\end{equation*}

\subsection{Results}

\subsection{Conclusion}
By performing some simplifying assumptions on the reaction \ref{eq:full}, we
have arrived at a term for the rate of RNAP elongation that is similar to the
term we originally started with.

The biggest difference between the two expressions is in the sign of $\Delta
G_{\text{RNA-DNA}}$. Since values for $Keq$ are around 1, taking the logarithm
does not change these values much in magnitude ($log(x)$ is near linear around
$x = 1$).

As a result, I simulated a set of ordinary differential equations based on
equation \ref{eq:stuff}. I get a nice correlation $(r = 0.7)$ -- except that I
must change the sign of $\Delta G_{\text{RNA-DNA}}$ from negative to positive.
Otherwise the correlation breaks down.

What is the cause of the sign-mismatch? I have gone over the maths several
times and find no mistakes. Is there a way around this which makes sense
physically?

% new from biblatex
%\printbibliography

\end{document}


