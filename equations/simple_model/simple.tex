%\documentclass[a4paper, draft]{article}
\documentclass[a4paper]{article}

% A new package for bibliography
% Use biber for bibtex and pdflatex instead of latex
\usepackage[backend=biber]{biblatex}
% new from biblatex
\addbibresource{/home/jorgsk/phdproject/bibtex/jorgsk.bib}

% Support for unicode
\usepackage[utf8]{inputenc}
\usepackage[T1]{fontenc}

\usepackage{amsmath,amsfonts,amssymb,amsthm,booktabs,array,mathtools,mhchem}
% consider package mhchem for typesetting chemical formulas

% Proper space and font for integral differential term
\newcommand{\te}[1]{\text{#1}} 
\newcommand{\dd}{\; \mathrm{d}} 
% Shorcut for ODEs with proper font
\newcommand{\diff}[2]{\frac{\mathrm{d} #1}{\mathrm{d} #2}}
% Shortcut for PDEs with proper font (shortcut: PDB)
\newcommand{\pdiff}[2]{\frac{\partial #1}{\partial #2}}
\newcommand{\pdiffn}[3]{\frac{\partial^{#3} #1}{\partial #2^{#3}}}

% Absolute value and norm commands.
% Read the mathtools.pdf to fix these!
\providecommand{\abs}[1]{\lvert#1\rvert} 
\providecommand{\norm}[1]{\lVert#1\rVert}

% Set the depth of section numbering
\setcounter{secnumdepth}{0}

\begin{document} 
\noindent

\subsection{Background}
We have found a strong and consistent correlation $(r ~ 0.8, p < 0.0001)$
between the productive yield (PY) and the parameters $\Delta
G_{\text{RNA-DNA}}$ and $Keq$.  These are respectively the free energy of
RNA-DNA hybrid and the equilibrium constant for the movement of RNAP between
the post-translocated and pre-translocated states:
\\
\begin{align}
\ce{R^{\text{e}}
<=>
R^{\text{s}}
}
\end{align}
The correlation we have found is on this form:
\\
\begin{equation*}
	PY \sim e^{\Delta G_{\text{RNA-DNA}} - Keq}
\end{equation*}
\\
In plain text, the productive yield of a promoter correlates with the exponent
of the RNA-DNA free energy minus the equilibrium constant for translocation.
In the actual calculation, the two terms are calculated for each dinucleotide
of the initially transcribed sequence (ITS) of a promoter.

$\text{RNA-DNA}$ and $Keq$ are known, or assumed to be involved in RNAP translocation,
and therefore this correlation makes sense. We would like to say that a
combination of RNA-DNA free energy and translocation explain the ITS-specific
abortive trascription rates. The challenge in this report is to find a physical
motivation for the term
\begin{equation*}
	e^{\Delta G_{\text{RNA-DNA}} - Keq}
\end{equation*}

\subsection{Chemical equation for RNAP movement}
\subsection{Chemical equation for RNAP movement}

Below is a reaction equation for the RNA polymerase as it is translocating on DNA.

\begin{itemize}
	\item $\te{R}_i^{\te{e}}$ is the pre-translocated polymerase at nucleotide $i$
	\item $\te{R}_{i+1}^{\te{e}}$ is the pre-translocated polymerase at nucleotide $i+1$
	\item $\te{R}_i^{\te{s}}$ is the post-translocated polymerase at nucleotide $i$
	\item $\te{N}_{\te{t}}$ is the incoming NTP
	\item $\te{PP}_i$ is the released $\te{PP}_i$ from nucleotide incorporation
\end{itemize}
In the reaction equation RNAP osscillates between the pre-translocated and the
post-translocated step at some nucleotide position $i$. In the
post-translocated step, a nucleotide $\te{N}_{\te{t}}$ may bind in the active
site. This nucleotide might then be incorporated, bringing RNAP to position
$i+1$ and at the same time releasing $\te{PP}_i$.
\\
\begin{align}
\dots\ce{R_i^{\text{e}}
<=>
R_i^{\text{s}}
<=>
R_i^{\text{s}}N_t
<=>
R_{i+1}^{\text{e}}
+ PP_i\dots
}
\label{eq:full}
\end{align}
\\
I wish to simplity equation \ref{eq:full} in three steps. In the first step, I
ignore the concentration of $\te{PP}_i$ and assume that the nucleotide
incorporation step is irreversible.
\\
\begin{align}
\dots\ce{R_i^{\text{e}}
<=>
R_i^{\text{s}}
<=>
R_i^{\text{s}}N_t
->
R_{i+1}^{\text{e}}\dots
}
\label{eq:reduced1}
\end{align}
\\
In the second step, we skip the nucleotide binding step, assuming this
step not to be rate-limiting or sequence dependent.
\\
\begin{align}
\dots\ce{R_i^{\text{e}}
<=>
R_i^{\text{s}}
->
R_{i+1}^{\text{e}}\dots
}
\label{eq:reduced2}
\end{align}
\\
In the third step, we assume that the osscillation between post-translocation and
pre-translocation steps is rapid compared to nucleotide binding and catalysis.
I can then realte the relationship between the pre- and post-translocated
states by the equilibrium constant for that reaction ($Keq_i$):
\begin{equation}
	\te{R}_i^{\te{s}} = \frac{\te{R}_i^{\te{e}}}{Keq_i}
	\label{eq:keq}
\end{equation}
I also assign the reaction rate $k_1$ to the nucleotide incorporation step:
\\
\begin{align}
\dots\ce{R_i^{\text{e}}
<=>[\ce{Keq_i}]
R_i^{\text{s}}
->[\ce{k1_i}]
R_{i+1}^{\text{e}}\dots
}
\label{eq:reduced3}
\end{align}
\\
\subsection{From chemical equation to differential equation}
At this stage, we would like to transfer the reaction equation into differential
equations which represent rate of change of RNAP at the different positions.
The expression for $\te{R}_{i+1}^{\te{e}}$ is according to mass action kinetics
\\
\begin{equation*}
	\diff{\te{R}_{i+1}^{\te{e}}}{t} = k_1\te{R}_{i}^{\te{s}}
\end{equation*}
\\
By inserting the relationship from equation \ref{eq:keq}, we get the following
final expression
\\
\begin{equation}
	\diff{\te{R}_{i+1}^{\te{e}}}{t} = \frac{k_1}{Keq_i}\te{R}_{i}^{\te{e}}
	\label{eq:differ}
\end{equation}
\\
This equation relates the rate of change of RNAP at position $i+1$ to the
amount of RNAP at position $i$.

\subsubsection{The reaction constant $k_1$}
We write the reaction constant $k_1$ as a special form of the Arrhenius
equation:
\begin{equation*}
	k_1 = Ke^{-\frac{\Delta G}{RT}}
\end{equation*}
Here, $\Delta G$ is the Gibbs free energy for the reaction, $R$ is the
universal gas constant and $T$ is the temperature in Kelvin.

To evaluate the term
\begin{equation*}
	\frac{k_1}{Keq}
\end{equation*}
from equation \ref{eq:differ} (without the index $i$ for simplicity), we write
\begin{equation*}
	Keq = e^{\te{ln}(Keq)},
\end{equation*}
where ln is the natural logarithm. We insert this into $\frac{k_1}{Keq}$ to get
\begin{align*}
	\frac{k_1}{Keq} &= \frac{Ke^{-\frac{\Delta G}{RT}}}{e^{\te{ln}(Keq)}}\\
	                &= Ke^{-\frac{\Delta G}{RT} - \te{ln}(Keq)}
\end{align*}

Which is close to what we want to get, but the sign is wrong!

% new from biblatex
%\printbibliography

\end{document}


